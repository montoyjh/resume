\documentclass[a4paper,10pt]{article}

%A Few Useful Packages
\usepackage{marvosym}
\usepackage{fontspec} 					%for loading fonts
\usepackage{xunicode,xltxtra,url,parskip} 	%other packages for formatting
\RequirePackage{color,graphicx}
\usepackage[usenames,dvipsnames]{xcolor}
\usepackage[big]{layaureo} 				%better formatting of the A4 page
% an alternative to Layaureo can be ** \usepackage{fullpage} **
\usepackage{supertabular} 				%for Grades
\usepackage{titlesec}					%custom \section
\usepackage[sorting=ydnt,
            maxbibnames=99,
            citestyle=authoryear,
            doi=false,
            isbn=false,
            url=false]{biblatex}
\renewbibmacro{in:}{}
%\usepackage{natbib}
%Setup hyperref package, and colours for links
\usepackage{hyperref}
\definecolor{linkcolour}{rgb}{0,0.2,0.6}
\hypersetup{colorlinks,breaklinks,urlcolor=linkcolour, linkcolor=linkcolour}

% bib section
\DefineBibliographyStrings{english}{%
  references = {Publications},
  }
\addbibresource{cv.bib}

\newcommand{\makeauthorbold}[1]{%
  \DeclareNameFormat{author}{%
    \ifthenelse{\value{listcount}=1}
    {%
      {\expandafter\ifstrequal\expandafter{\namepartfamily}{#1}{\mkbibbold{\namepartfamily\addcomma\addspace \namepartgiveni}}{\namepartfamily\addcomma\addspace \namepartgiveni}}
      %
    }{\ifnumless{\value{listcount}}{\value{liststop}}
        {\expandafter\ifstrequal\expandafter{\namepartfamily}{#1}{\mkbibbold{\addcomma\addspace \namepartfamily\addcomma\addspace \namepartgiveni}}{\addcomma\addspace \namepartfamily\addcomma\addspace \namepartgiveni}}
        {\expandafter\ifstrequal\expandafter{\namepartfamily}{#1}{\mkbibbold{\addcomma\addspace \namepartfamily\addcomma\addspace \namepartgiveni\addcomma\isdot}}{\addcomma\addspace \namepartfamily\addcomma\addspace \namepartgiveni\addcomma\isdot}}%
      }
    \ifthenelse{\value{listcount}<\value{liststop}}
    {\addcomma\space}
  }
}


% goofy shit to fix sub/sup
\makeatletter
% get rid of the closing braces
\def\textless{\afterassignment\textless@\let\next= }
% get the tag type
\def\textless@#1#{\@nameuse{textless@#1}}
% code for <sub>
\def\textless@sub#1#2/sub#3{%
  \ensuremath{_{\let\textless\relax#2}}%
  \egroup % matches the first brace
}
% code for <sub>
\def\textless@sup#1#2/sup#3{%
  \ensuremath{^{\let\textless\relax#2}}%
  \egroup % matches the first brace
}
\makeatother

\makeauthorbold{Montoya}

\setlength{\bibitemsep}{0.2\baselineskip}
%\DeclareFieldFormat{labelnumberwidth}{}
%\setlength{\biblabelsep}{0pt}
%FONTS
\defaultfontfeatures{Mapping=tex-text}
%\setmainfont[SmallCapsFont = Fontin SmallCaps]{Fontin}
%%% modified for Karol Kozioł for ShareLaTeX use
\setmainfont[
SmallCapsFont = Fontin-SmallCaps.otf,
BoldFont = Fontin-Bold.otf,
ItalicFont = Fontin-Italic.otf
]
{Fontin.otf}
%%%

%CV Sections inspired by: 
%http://stefano.italians.nl/archives/26
\titleformat{\section}{\Large\scshape\raggedright}{}{0em}{}[\titlerule]
\titlespacing{\section}{0pt}{3pt}{3pt}
%Tweak a bit the top margin
%\addtolength{\voffset}{-1.3cm}

%Italian hyphenation for the word: ''corporations''
\hyphenation{im-pre-se}

%-------------WATERMARK TEST [**not part of a CV**]---------------
\usepackage[absolute]{textpos}

\setlength{\TPHorizModule}{30mm}
\setlength{\TPVertModule}{\TPHorizModule}
\textblockorigin{2mm}{0.65\paperheight}
\setlength{\parindent}{0pt}

%--------------------BEGIN DOCUMENT----------------------
\begin{document}

%WATERMARK TEST [**not part of a CV**]---------------
%\font\wm=''Baskerville:color=787878'' at 8pt
%\font\wmweb=''Baskerville:color=FF1493'' at 8pt
%{\wm 
%	\begin{textblock}{1}(0,0)
%		\rotatebox{-90}{\parbox{500mm}{
%			Typeset by Alessandro Plasmati with \XeTeX\  \today\ for 
%			{\wmweb \href{http://www.aleplasmati.comuv.com}{aleplasmati.comuv.com}}
%		}
%	}
%	\end{textblock}
%}

\pagestyle{empty} % non-numbered pages

\font\fb=''[cmr10]'' %for use with \LaTeX command

%--------------------TITLE-------------
\par{\centering
		{\Huge Joseph \textsc{Montoya}
	}\bigskip\par}

%--------------------SECTIONS-----------------------------------
%Section: Personal Data
%\section{Personal Data}

\begin{tabular}{rl}
    %\textsc{Place and Date of Birth:} & Someplace, Italy  | dd Month 1912 \\
    \textsc{Address:}   & 1 Cyclotron Road, M/S 33R0146, Berkeley, CA 94703 \\
    \textsc{Phone:}     & (843) 276-1397\\
    \textsc{email:}     & \href{mailto:montoyjh@lbl.gov}{montoyjh@lbl.gov}
\end{tabular}

%Section: Education
\section{Education}
\begin{tabular}{ll}	
2015 & Ph.D. \textsc{Chemical Engineering}, \textbf{Stanford University}, Stanford, CA\\
& Thesis: ``Theoretical electrocatalysis for renewable fuels and chemicals'' \vspace{0.1in} \\
2014 & M.S. \textsc{Chemical Engineering}, \textbf{Stanford University}, Stanford, CA \vspace{0.1in} \\
2010 & B.S. \textsc{Chemical Engineering}, \textbf{University of South Carolina}, Columbia, SC\\
2010 & B.S. \textsc{Mathematics}, \textbf{University of South Carolina}, Columbia, SC\\
2010 & Certificate, \textsc{Piano Performance}, \textbf{University of South Carolina}, Columbia SC
\end{tabular}
\vspace{0.2in}
%Section: Work Experience at the top
\section{Professional History}
\begin{tabular}{l|p{11cm}}
 \emph{2015-Current} & Postdoctoral Researcher, \textbf{Lawrence Berkeley National Laboratory} \\
 &\emph{Advisor: Prof. Kristin \textsc{Persson}, Energy Technologies Area}\\& \smallskip
 \small{Part of the core development team for the Materials Project.  Projects include performing 
 high-throughput calculations of elastic tensors, developing workflows for surface science, refining approaches 
 to electrochemical stability analysis, and determining trends in complex oxide reactivity for solar water splitting
 and CO$_2$ reduction}\\\multicolumn{2}{c}{} \\
 \textsc{2010-2015} & Graduate Student Researcher, \textbf{Stanford University}\\&\emph{Advisor: Prof. Jens \textsc{N\o rskov}, Dept. of Chemical Engineering}\\&\small{Performed DFT simulations of the electrocatalysis of nitrogen electroreduction, total water splitting on perovskite oxides, and C-C coupling in CO$_2$ electroreduction.  Constructed linear scaling relations to identify trends and structure sensitivity of various catalytic surfaces.  Used and developed electrochemical kinetics in CatMAP software to model steady-state turnover and selectivity.  Created tools for managing data corresponding to over 5000 bulk compounds, 500 surfaces, and 10000 adsorbate configurations.}\\\multicolumn{2}{c}{} \\
\textsc{2006-2010} & Undergraduate Researcher, \textbf{University of South Carolina}\\&\emph{Advisors: John \textsc{Monnier}, Chris \textsc{Williams}, Dept. of Chemical Engineering}\\&\small{Conducted experimental synthesis and characterization of bimetallic catalysts for the selective hydrogenation of acetylene and epoxybutene.  Used electroless deposition (ED) technique to prepare [Cu,Ag,Au][Pd,Pt]/SiO$_2$ bimetallic catalysts.  Characterized catalyst surface area and activity.}\\ \multicolumn{2}{c}{} \\
\textsc{2008} & IREU, \textbf{King Mongkut University of Technology, Thonburi, Thailand}\\&\emph{Advisors: Preecha \textsc{Termsuksawad}, Dept. of Materials Science}\\&\small{Studied extraction of Ni-Co alloys from spent nickel metal-hydride (Ni-MH) batteries via electrochemical characterization with cyclic voltammetry (CV) and chronoamperometry.}
\end{tabular}
\vspace{0.2in}
%Section: Scholarships and additional info
\section{Selected Awards and Honors}
\begin{tabular}{ll}
2013 & Richard J. Kokes Award, \textbf{North American Catalysis Society} \\
2010 & Graduate Research Fellowship, \textbf{National Science Foundation} \\
2010 & Finalist, \textbf{Rhodes Scholarship} \\
2010 & Outstanding Undergraduate Student in Mathematics, \textbf{University of South Carolina} \\
2009 & Outstanding Chemical Engineering Senior, \textbf{University of South Carolina} \\
2008 & \textbf{Barry M. Goldwater Scholarship} \\
\end{tabular}

\begin{refsection}
%\section{Publications}
%\vspace{0.1in}
\nocite{Montoya2018,Singh2018,Singh2017,Dagdelen2017,Mathew2017,C7CP02855E,Montoya2017a,Montoya2017,Latimer2016,Seitz2016,Sandberg2016,Tsai2016,Wang2015,Bertheussen2016,Doyle2015,Montoya2015,Montoya2015a,Schaal2007,Montoya2014,Hansen2013,Montoya2013}
%\nocite{*}
%\bibliographystyle{unsrt}
\printbibliography
\end{refsection}
% \vspace{0.2in}
% \begin{refsection}
% %\subsection{Forthcoming publications}
% \nocite{Montoya2017b,Singh2017a}
% \printbibliography[title=Forthcoming Publications]
% \end{refsection}
% \vspace{0.2in}
\section{Recent Talks}
\begin{tabular}{ll}
March 2018 & \textsc{The Materials Project: A Science Gateway for Computational}\\
& \textsc{Materials Science} \vspace{0.04in} \\
& \textbf{ACS Spring Meeting, New Orleans, LA} \vspace{0.1in} \\
November 2017 & \textsc{A High-Throughput Computational Screening Approach for}\\
& \textsc{Solar Fuels Photoelectrocatalysis} \vspace{0.02in} \\
&  \textsc{New Milestones and Challenges to High-Throughput Computation} \\ 
& \textsc{of Elastic Properties on the Materials Project} \vspace{0.02in} \\
& \textsc{Climbing the Volcano: Active-Site Engineering at the Atomic Scale} \vspace{0.04in}\\
& \textbf{AIChE Annual Meeting, Minneapolis, MN} \vspace{0.1in} \\
September 2017 & \textsc{The Materials Project: Challenges and Opportunities}\\
& \textsc{in High-throughput computational Materials Science} \vspace{0.04in} \\
& \textbf{E-CAM Industry Workshop, "From Atom to Material", Cambridge, UK} \vspace{0.1in} \\
August 2017 & \textsc{Towards a solar fuels future: Theoretical metrics for} \\
& \textsc{photoelectrocatalyst screening} \vspace{0.02in} \\
& \textsc{High-throughput workflows for determining} \\
& \textsc{adsorption energies on solid surfaces} \vspace{0.04in} \\
& \textbf{ACS Fall Meeting, Washington, DC} \vspace{0.1in} \\
June 2017 &\textsc{The Materials Project: Challenges and Opportunities}\\
& \textsc{in High-throughput computational Materials Science} \vspace{0.04in} \\
& \textbf{ECS Annual Meeting, New Orleans, LA} \vspace{0.06in} \\
May 2017 & \textsc{Electrocatalysis for renewable energy via}\\
& \textsc{active-site engineering at the atomic scale}\vspace{0.04in}\\
& \textbf{Seminar, University of South Carolina, Columbia, SC}\\
\end{tabular}

\section{Teaching experience}
\begin{tabular}{ll}
  \textbf{Teaching assistant}
 & Served twice as TA to Prof. Eric Shaqfeh in ``Applied  \\
  Stanford University &Mathemetics for the Chemical and Biological Sciencies,'' \\
 Fall 2012-2013 &an introductory graduate course for chemical engineers. \vspace{0.05in} \\
 &Awarded ``Outstanding Chemical Engineering TA'' in 2013\vspace{0.13in}\\

  \textbf{TA mentor}
 & Selected and served as a mentor in the TA Mentorship program \\
  Stanford University & led by Senior Lecturer Lisa Hwang. Led training, formulated  \\
 Fall 2014-2015 & teaching goals, and offered assessment to first  \\
& and second-year TAs \vspace{0.13in}\\

 \textbf{Software Carpentry}
  &Certified Software Carpentry instructor for workshops on  \\
 January 2016-Present &introductory and advanced python for scientific computing. \vspace{0.05in} \\
 &Served as lead or co-instructor in six eight-hour workshops \vspace{0.05in} \\ 
 &Served as secondary instructor/helper in in five workshops. \vspace{0.05in} \\
 &Developed coursework for \href{https://swcarpentry.github.io/python-second-language/}{``Python as a Second Language"} \\
 & and \href{https://swcarpentry.github.io/python-novice-gapminder/}{``Introduction to Python with Gapminder Data"}\vspace{0.13in} \\ 

\textbf{The Materials Project} & Instructor and course designer in the annual \\
Fall 2016-2017 & \href{https://github.com/materialsproject/workshop-2017}{Materials Project Workshop}. \vspace{0.13in}\\

\textbf{Guest Lecturer} & Two guest lectures on pymatgen and the materials project, \\ 
UC Berkeley & in MSE 215: ``Computational Materials Science'' \\
Fall 2016-2017 & 
 \end{tabular} \vspace{0.2in}

\section{Technical Skills and Background}
\begin{tabular}{ll}
 %%%
 \textbf{High-performance} & Extensive experience in python software development for \\ 
 \textbf{computing}  & scientific applications.  Experience using NERSC, Oak Ridge (OLCF), \\
 & Argonne (ALCF), and SLAC supercomputing Resources.  Extensive  \\
 &experience in automated workflow management.\vspace{0.13in} \\

 \textbf{Software development} & Fluent in python.  Experience in C++, MATLAB, and Mathematica. \vspace{0.05in} \\
 & Contributor to \href{pymatgen.org}{pymatgen}, \href{https://materialsproject.github.io/fireworks/}{FireWorks}, \href{https://hackingmaterials.github.io/atomate}{atomate}, and \href{http://catmap.readthedocs.io/en/latest/}{CatMAP}\\
 & open source projects. See \href{https://github.com/montoyjh}{my github profile}.  \vspace{0.13in}\\


 \textbf{Data science} & Extensive experience in MongoDB infrastructure for materials \\
 & data management at the \href{http://materialsproject.org}{materials project}.  Experience \\
 & using tensorflow, scikit-learn, pandas, jupyter and \\
 & numerous other python libraries for data science.\vspace{0.13in}\\
%%%
 \textbf{DFT Simulations} & Extensive experience using density functional theory software, \\
 & including VASP, Quantum Espresso, GPAW, and Dacapo \vspace{0.13in}\\
 \textbf{Technical writing} & Experience in preparation of scientific manuscripts, research \\
 \textbf{and reviewing} & proposals, \LaTeX, and data visualization \vspace{0.05in}\\
& Reviewed articles in ACS Catalysis (6), J. Phys. Chem. (1), \\
& Chem. Mater. (6), and Catalysis Today (2), Nat. Catalysis (1)
 \end{tabular}
\section{Hobbies and other interests}
\begin{tabular}{ll}
 \textsc{Music} & Advanced proficiency in jazz vibraphone, completed coursework in jazz \\
 & improvisation, jazz standards for gigs, and current member of \textit{Music of the} \\
 & \textit{Masters} ensemble at California Jazz Conservatory's Community Jazz School.\vspace{0.05in}\\
 & Collegiate proficiency in classical piano.  Earned a performance certificate\\
 & from two recitals given while at University of South Carolina in the studio of\\
 & Prof. Joseph Rackers.\vspace{0.05in}\\
 & Intermediate proficiency in steel drumming, 3 years experience in Palmetto \\
 & Pans Steel Drum ensemble. \vspace{0.05in}\\
  \textsc{Quiz Bowl} & Experienced moderator, player, and question writer for ACF and NAQT \\
 & format team academic buzzer-based competitions, formerly for \\
 & University of South Carolina and Stanford, currently as a volunteer. \vspace{0.05in} \\
\textsc{Running} & Participated in various 5K, 10K, and half-marathon length races.
\end{tabular}
\nocite{*}
\end{document}
